\chapter{Verwandte Arbeiten}

Es gibt im näheren Kontext dieses Projekts vier Veröffentlichungen, die im Folgenden vorgestellt und eingeordnet werden.

Das erste Paper stellt das Framework ViSTA zur Szenario-Generierung vor, welches sowohl automatische (auf Basis zufälligen Parametern) und manuelle Testszenarien zulässt \cite{ViSTA}. Dieses Framework fokussiert sich auf eine Testausführung mit SVL und Apollo und ist deshalb für dieses Projekt nicht relevant.

Auch die zweite Veröffentlichung stellt unter dem Titel \enquote{ISS-Scenario} ein eigenes Test-Framework vor, das die Ausführung der Testfälle wird mit CARLA realisiert \cite{li2024issscenarioscenariobasedtestingcarla}. Das Paper stellt ein parametrisierten Szenario-Bibliothek vor und zeigt ein Diagramm dazu, dass mit der vorgestellten Methode Kollisionen um ein Vielfaches häufiger provoziert werden können als bei anderen Methoden. Diese Werte wurden allerdings nicht von Dritten bestätigt und das Paper liegt nur als Preprint vor. Daher und da die Simulation mit CARLA den Rahmen dieses Projekts überschreitet, hat dieses Paper keine weitere Bedeutung für das Projekt.

Die Publikation \enquote{Generating Critical Test Scenarios for Autonomous Driving Systems via Influential Behavior Patterns} verfolgt ein ähnliches Ziel wie \enquote{ISS-Scenario} mit \enquote{AVASTRA} -- einem Reinforcement Learning Ansatz, der Szenarien bildet, die das zu testende Fahrzeug in gefährliche Situation bringen \cite{GeneratingCritialTestScenarios}. Dieses Paper zeigt, dass RL als effektives Werkzeug bei der Entwicklung von Test-Szenarien eingesetzt werden kann.

In einer Studie mit dem Titel \enquote{Testing of autonomous driving systems: where are we and where should we go?} wurde die Frage untersucht, welche Praktiken aktuell im Testen autonomer Fahrzeuge angewandt werden und welche Unterstützung sich Entwickler dieser Tests wünschen \cite{WhereShouldWeGo}. Die Studie kommt zu dem Ergebnis, dass die Entwickler aktuell keine ausreichende Unterstützung bei der Analyse von Fahrzeugdaten durch geeignete Werkzeuge haben.
