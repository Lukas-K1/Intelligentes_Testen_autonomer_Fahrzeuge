\chapter{Motivation}
\label{chap:motivation}
Autonomes Fahren ist in der heutigen Zeit keine ferne Vision mehr, sondern bereits Realität. Unternehmen wie Waymo in San Francisco \cite{WaymoRobotaxis2025}, Baidu in Wuhan \cite{BaiduRobotaxi2024} oder Volkswagen in Hamburg \cite{VolkswagenRobotaxis2025} zeigen, dass fahrerlose Fahrzeuge am Straßenverkehr teilnehmen können. Hierdurch wachsen jedoch auch die Herausforderungen, die Sicherheit solcher Systeme zu gewährleisten und dementsprechend zu validieren. Die Vielzahl potenzieller Verkehrssituationen ist praktisch unbegrenzt, von einfachen Überholmanövern bis hin zu komplexen Interaktionen mit Fußgängern, Radfahrern, Ampeln und mehreren Fahrzeugen gleichzeitig. Hinzu kommen länderspezifische Verkehrsregeln und Regularien, die eine vollständige Erprobung im Realverkehr nahezu unmöglich machen.

Simulationen stellen daher einen unverzichtbaren Baustein in der Validierung autonomer Fahrfunktionen dar, da reale Testfälle nicht in der notwendigen Breite und Tiefe skalierbar sind. Mit Standards wie ASAM OpenScenario existieren bereits Ansätze, Szenarien in strukturierter Form zu beschreiben. Allerdings sind die XML-basierten Szenarien sehr aufwändig zu modellieren und müssen für jedes Testziel individuell erstellt werden. Neuere domänenspezifische Sprachen, wie die auf Constraints basierende OpenScenario DSL, versprechen zwar leichtere Handhabung, sind aber noch nicht weit verbreitet und erfordern ebenfalls manuelle Anpassungen. Industrielle Lösungen wie Fortellix Foretify nutzen Constraint-Solver und Planungsalgorithmen, erweisen sich in der Praxis jedoch häufig als starr und schwer automatisierbar.

Vor diesem Hintergrund ist die Entwicklung neuer, flexiblerer Ansätze zu automatisierten Szenariengenerierung von zentraler Bedeutung. Behavioral Programming bietet hierfür eine attraktive Grundlage. Es erlaubt eine modulare Modellierung von Szenarien und bildet konkurrierende Verhaltensweisen auf natürliche Weise ab. Ergänzend eröffnet Reinforcement Learning die Möglichkeit, aus abstrakten Szenariobeschreibungen, etwa ein Auto überholt das VUT und bremst anschließend, konkrete, ausführbare Szenarien zu generieren, die den abstrakten Vorgaben genügen.

Dieses Projekt verfolgt das Ziel, die Kombination von Behavioral Programming und Reinforcement Learning als neuartigen Ansatz zur automatisierten Szenariengenerierung zu untersuchen. Dabei sollen nicht nur Methoden zur Modellierung abstrakter und konkreter Szenarien entwickelt werden, sondern auch Verfahren, wie die ausgeführten Szenarien geloggt und visuell aufbereitet werden können, um Test-Engineers eine transparente Analyse zu ermöglichen. Vor diesem Hintergrund ergibt sich die zentrale Forschungsfrage:

\begin{quote}
\paragraph{Wie kann BPpy zur Modellierung ausführbarer Testszenarien im autonomen Fahren eingesetzt werden?}
\end{quote}

Zur Beantwortung dieser Frage werden folgende Unterfragen untersucht:
\begin{itemize}
    \item Wie können Szenarien in BPpy modelliert werden und welche grundsätzlichen Unterschiede bestehen im Vergleich zu OpenSCENARIO?
    \item Welche Möglichkeiten der Komposition bietet BPpy für Szenarien, etwa sequenziell oder parallel?
    \item Wie kann Reinforcement Learning eingesetzt werden, um einen Agenten zu trainieren, der in der Lage ist Testszenarien so zu steuern, dass diese abstrakten Szenarien genügen?
    \item Wie können ausgeführte Szenarien geloggt und visualisiert werden, um eine transparente Analyse zu ermöglichen?
\end{itemize}

Das Vorhaben versteht sich als Machbarkeitsstudie. Es geht nicht darum, ein industrietaugliches Framework zu entwickeln, sondern vielmehr um die grundlegende Frage, ob sich durch die Kombination von Behavioral Programming und Reinforcement Learning ein flexibler, automatisierbarer Ansatz zur Generierung konkreter Testszenarien aus abstrakten Vorgaben realisieren lässt.