\chapter{Ausblick}

Die im Projektverlauf gewonnenen Erkenntnisse zeigen eindrucksvoll, welches Potenzial die Kombination aus Behavioral Programming, Reinforcement Learning und flexibler Simulation für die Entwicklung und Überprüfung autonomer Fahrsysteme bietet\cite{EinfuehrungIntegrationsprojekt}. Auch wenn dieses Vorhaben im Rahmen eines Masterstudiums stattfand und nicht auf eine industrielle Anwendung abzielt, konnten zentrale Methoden auf Praxistauglichkeit überprüft und wichtige Grundlagen vermittelt werden.

Rückblickend eröffnen sich aus unserer studentischen Perspektive verschiedene Möglichkeiten für zukünftige Arbeiten und Vertiefungen:
\begin{itemize}
    \item \textbf{Komplexere Szenarien und Umgebungen:} In folgenden Projekten könnten noch anspruchsvollere Szenarientypen, etwa mit Kreuzungen, wetterspezifischen Einflüssen oder besonderen Fahrmanövern, modelliert werden. Die Ausgestaltung realistischer Verkehrsabläufe bleibt ein spannendes und lehrreiches Forschungsfeld.
    \item \textbf{Automatisierung der Szenarienerstellung:} Es wäre erstrebenswert, die gesamte Generierung von Testszenarien weiter zu automatisieren und deren Konsistenz zu verbessern. Dies würde nicht nur den Projektaufwand verringern, sondern auch die Aussagekraft von Simulationen deutlich steigern.
    \item \textbf{Weiterentwicklung von KI-Steuerung und RL:} Die Verbindung aus Reinforcement Learning und modularen Szenario-Architekturen sollte weiter vertieft werden, um robuste und vielseitige KI-Agenten zu entwickeln. Insbesondere das Belohnungsdesign und die Kopplung an geeignete Simulationsdaten bieten hier viel Potenzial für weitere Experimente und studentische Forschung.
    \item \textbf{Vertiefung von Visualisierung und Auswertung:} Die bereits umgesetzten Dashboards und Analysewerkzeuge können in kommenden Arbeiten noch ausgebaut und benutzerfreundlicher gestaltet werden. Eine verständliche und transparente Darstellung der Ergebnisse fördert das Verständnis – nicht nur für uns, sondern auch für andere Studierende.
\end{itemize}

Das Projekt hat uns gezeigt, wie wichtig eine interdisziplinäre Herangehensweise aus Informatik, Simulation und methodischer Auswertung für die Lösungsentwicklung ist. Für die weitere persönliche und wissenschaftliche Entwicklung bieten die erarbeiteten Grundlagen vielfältige Ansatzpunkte zur Vertiefung — sowohl im Rahmen von Abschlussarbeiten als auch im Team oder in praktischen Workshops.
