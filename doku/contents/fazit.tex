\chapter{Fazit}

Dieses Projekt hat innovative Methoden zur Simulation und zum Testen autonomer Fahrsysteme erfolgreich erforscht. Die Kombination von Behavioral Programming mit BPpy und den Simulationsumgebungen SUMO und HighwayEnv ermöglichte sowohl abstrakte als auch konkrete Szenarien modellieren und evaluieren zu können.

\paragraph{Modulare Szenarienmodellierung}
BPpy zeigte sich als sehr geeignet, um komplexe, konkurrierende Verhaltensaspekte modular zu spezifizieren. Die Nutzung von B-Threads ermöglicht eine flexible, inkrementelle Erstellung von Testszenarien mit hoher Wiederverwendbarkeit.

\paragraph{Stärken der Simulationsplattformen}
\begin{itemize}
    \item \textbf{SUMO} überzeugte durch realistische Karten und Verkehrsabläufe, insbesondere für komplexe Praxisszenarien.
    \item \textbf{HighwayEnv} ist als effiziente Prototyping-Plattform ideal für erstes RL-Training, geriet aber bei hoher Komplexität an Grenzen.
\end{itemize}

\paragraph{Erfahrungen mit Reinforcement Learning}
Das Zusammenspiel von RL-Agenten und BPpy bietet ein großes Potenzial, stellt aber hohe Anforderungen an Belohnungsdesign und Schnittstellenkonsistenz. Erste Implementierungen zeigten Wege zur episodischen Szenariosteuerung, machten aber auch den Forschungsbedarf deutlich.

\paragraph{Logging und Reporting}
Die systematische Erfassung und Auswertung von Simulationsdaten führte zu transparenten, nutzerfreundlichen Visualisierungen und Dashboards. Dies unterstützt eine zielgerichtete Analyse und Fehleridentifikation im Testprozess.

\paragraph{Herausforderungen und offene Punkte}
\begin{itemize}
    \item Heterogene Simulationsschnittstellen und mangelhafte Dokumentation forderten viel methodische Flexibilität.
    \item Die Skalierbarkeit komplexer Szenarien und das Belohnungsengineering sind weiterhin anspruchsvolle Aufgaben.
    \item Erweiterungen bezüglich Umwelteinflüssen und fortgeschrittener RL-Anbindung sind dringend erforderlich.
\end{itemize}

Zusammenfassend zeigt dieses Projekt, dass die Verbindung moderner KI-Methoden, modularer Softwarearchitektur und realitätsnaher Simulation eine solide Basis für zukünftige Testverfahren autonomer Fahrfunktionen bietet. Die gewonnenen Erkenntnisse können wertvolle Impulse für weiterführende Forschung und praktische Umsetzungen liefern.

Der Weg zur vollständigen Automatisierung und Stabilisierung der Szenariengeneration sowie die Verbesserung der Nutzerfreundlichkeit bleiben zentrale Aufgaben künftiger Entwicklungsarbeit.
